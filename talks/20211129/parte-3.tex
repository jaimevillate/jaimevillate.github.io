\documentclass{beamer}
\usetheme{Madrid}
\usepackage[portuguese]{babel}
\usepackage[T1]{fontenc}
\usepackage{dejavu,tgbonum,graphicx,xcolor}
\setlength{\parindent}{0pt}
\setlength{\parskip}{0.7em}
\definecolor{brown}{rgb}{0.75,0.55,0.1}
\definecolor{violet}{rgb}{0.7,0.1,0.8}
\definecolor{blue}{rgb}{0,0,0.9}
\definecolor{green}{rgb}{0,0.6,0}
\hypersetup{
  colorlinks=true,
  linkcolor=cyan,
  citecolor=magenta,
  urlcolor=green,
  pdftitle={Introdução ao LaTeX, terceira parte},
  pdfauthor={Jaime E. Villate},
  pdfsubject={Latex},
  pdfkeywords={Latex, Tex},
}
\urlstyle{same}

\title{Introdução ao \LaTeX\ -- Terceira parte}
\author{Jaime Villate}
\institute[FEUP]{Faculdade de Engenharia da Universidade do Porto}
\date[13/12/2021]{13 de dezembro de 2021}

\begin{document}
\frame{\titlepage}
\begin{frame}
\frametitle{Conteúdo}
\tableofcontents
\end{frame}
\section{Tipos de letra}
\begin{frame}
\frametitle{Tipos de letra}
O tipo de letra muda-se usando:
\{\textcolor{blue}{$\backslash$estilo1$\backslash$estilo2\ldots} Texto\}
em que estilo pode ser:\pause
\begin{description}
\item[$\backslash$rmfamily] {\rmfamily Letra serifada.}\pause
\item[$\backslash$sffamily] {\sffamily Letra não serifada.}\pause
\item[$\backslash$ttfamily] {\ttfamily Tamanho uniforme.}\pause
\item[$\backslash$mdseries] {\mdseries Normal.}\pause
\item[$\backslash$bfseries] {\bfseries Negrita.}\pause
\item[$\backslash$upshape] {\upshape Normal.}\pause
\item[$\backslash$itshape] {\rmfamily\itshape Itálica.}\pause
\item[$\backslash$slshape] {\slshape Inclinada.}\pause
\item[$\backslash$scshape] {\rmfamily\scshape Pequenas maiúsculas.}\pause
\end{description}

O estilo do texto por omissão muda-se com:
\textcolor{blue}{$\backslash$renewcommand}\{\textcolor{blue}{$\backslash$familydefault}\}\{\textcolor{blue}{$\backslash$sfdefault}\} (ou rmdefault).

\end{frame}
\begin{frame}
\frametitle{Tamanho das letras}
O tamanho das letras altera-se usando:
\{\textcolor{blue}{$\backslash$tamanho} Texto\}\pause

Os possíveis tamanhos são:\pause
\begin{description}
\item[\phantom{xxxxxxiii}$\backslash$tiny] {\tiny Texto exemplo.}\pause
\item[\phantom{xxi}$\backslash$scriptsize] {\scriptsize Texto exemplo.}\pause
\item[$\backslash$footnotesize] {\footnotesize Texto exemplo.}\pause
\item[\phantom{xxxxxx}$\backslash$small] {\small Texto exemplo.}\pause
\item[\phantom{xi}$\backslash$normalsize] {\normalsize Texto exemplo.}\pause
\item[\phantom{xxxxxxi}$\backslash$large] {\large Texto exemplo.}\pause
\item[\phantom{xxxxxx}$\backslash$Large] {\Large Texto exemplo.}\pause
\item[\phantom{xxxxx}$\backslash$LARGE] {\LARGE Texto exemplo.}\pause
\item[\phantom{xxxxxxi}$\backslash$huge] {\huge Texto exemplo.}\pause
\item[\phantom{xxxxxxi}$\backslash$Huge] {\Huge Texto exemplo.}
\end{description}
\end{frame}
\begin{frame}
\frametitle{Letras para PDF}
A fonte original do \TeX\ (\emph{Computer Modern}), foi criada para
impressoras, antes de existir o formato PDF.

Hoje em dia é melhor usar fontes PDF, por meio do pacote:

\qquad\textcolor{violet}{$\backslash$usepackage}[\textcolor{brown}{T1}]\{\textcolor{blue}{fontenc}\}\pause

Existem muitas fontes disponíveis. Há um catálogo em:

\qquad\url{https://tug.org/FontCatalogue/}
\end{frame}
\section{Cabeçalhos}
\begin{frame}
\frametitle{Cabeçalhos}
Definem-se com o comando
\textcolor{violet}{$\backslash$pagestyle}\{\textcolor{blue}{estilo}\},
em que estilo pode ser:\pause
\begin{description}
\item[\phantom{xxxxoiii}plain] Número centrado, no fim da página.\pause
\item[\phantom{xxxxii}empty] Sem número.\pause
\item[\phantom{xxi}headings] Número no topo da página após cabeçalho.\pause
\item[myheadings] Semelhante a headings mas configurável.\pause
\end{description}

\textcolor{violet}{$\backslash$thispagestyle}\{estilo\} muda o estilo apenas
numa página.\pause

Para configurar os cabeçalhos, recomenda-se o pacote \textbf{fancyhdr}.
\end{frame}
\section{Notas de pé de pagina}
\begin{frame}
\frametitle{Notas de pé de pagina}
\textcolor{violet}{$\backslash$footnote}\{texto\} introduz uma nota de pé de
página.\pause

A nota é identificada por um número que aparecerá no lugar onde foi escrita a
nota. 
\end{frame}
\section{Tabelas}
\begin{frame}
\frametitle{Tabelas}
Colocadas num bloco aparte, num lugar apropriado do documento e numeradas
automaticamente. Definem-se com:
  
\qquad\textcolor{violet}{$\backslash$begin}\{\textcolor{blue}{table}\}conteúdo\textcolor{violet}{$\backslash$end}\{\textcolor{blue}{table}\}\pause

O \textbf{conteúdo} pode incluir:\pause
\begin{itemize}
\item O comando \textcolor{violet}{$\backslash$centering}, se quisermos
a tabela centrada.\\[12pt]\pause
\item Texto, gráficos ou um bloco \textbf{tabular}\\[12pt]\pause
\item O comando
\textcolor{violet}{$\backslash$caption}\{\textcolor{blue}{Legenda}\},
para incluir uma legenda.\\[12pt]\pause
\item O comando
\textcolor{violet}{$\backslash$label}\{\textcolor{blue}{nome}\}, para poder
fazer referência ao número da tabela.
\end{itemize}
\end{frame}
\begin{frame}
\frametitle{Blocos tabular}
Arranja o texto em linhas e colunas

\quad\textcolor{violet}{$\backslash$begin}\{\textcolor{blue}{tabular}\}\{\textbf{formato das colunas}\}\\
\qquad linha 1 coluna 1 \textcolor{red}{\&} linha 1 coluna 2\textcolor{red}{$\backslash$$\backslash$}\\
\qquad linha 2 coluna 1 \textcolor{red}{\&} linha 2 coluna 2\textcolor{red}{$\backslash$$\backslash$}\\
\quad\textcolor{violet}{$\backslash$end}\{\textcolor{blue}{tabular}\}\pause

O \textbf{formato das colunas} usa, por cada coluna, uma letra. Em
cada linha, as colunas separam-se com o carácter \textcolor{red}{\&} e
o fim da linha com \textcolor{red}{$\backslash$$\backslash$}\pause

Para colocar uma linha vertical, escreve-se uma barra vertical
\textcolor{red}{|} na definição do formato, antes e/ou depois da letra
que define a coluna.\pause

Para colocar uma linha horizontal usa-se o comando
\textcolor{blue}{$\backslash$hline} antes e/ou depois de uma linha.
\end{frame}
\begin{frame}
\frametitle{Formato das colunas num bloco tabular}
A letra para o formato de uma das colunas pode ser:\pause

\begin{description}
\item[\phantom{xxxxxxxxi}l] Alinhada à esquerda.\pause
\item[\phantom{xxxxxxxxi}r] Alinhada à direita.\pause
\item[\phantom{xxxxxxxxi}c] Centrada.\pause
\item[\phantom{i}p\{largura\}] Parágrafo alinhado no topo.\pause
\item[m\{largura\}] Parágrafo alinhado ao meio.\pause
\item[\phantom{i}b\{largura\}] Parágrafo alinhado por baixo.\pause
\end{description}

\textbf{Exemplo:} [\textcolor{blue}{|r|m\{2 cm\}|c|}]

Cria três colunas, separadas por barras verticais.

\end{frame}
\section{Índices}
\begin{frame}
\frametitle{Índices}
\textcolor{violet}{\bfseries $\backslash$tableofcontents}\\
\quad Cria automaticamente uma lista do conteúdo do\\
\quad documento: capítulos, secções, subsecções, etc.\pause

\textcolor{violet}{\bfseries $\backslash$listoffigures}\\
\quad Cria uma lista com as figuras no documento.\pause

\textcolor{violet}{\bfseries $\backslash$listoftables}\\
\quad Cria uma lista com as tabelas no documento.
\end{frame}
\section{Hiperligações}
\begin{frame}
\frametitle{Hiperligações}
\qquad\textcolor{violet}{$\backslash$usepackage}\{\textcolor{blue}{hyperref}\}

Cria hiperligações no ficheiro PDF e inclui informação sobre o
ficheiro PDF.\pause

Permite criar hiperligações para um endereço (URL) na Web com algum dos dois
comandos:\pause
\begin{itemize}
\item \textcolor{violet}{$\backslash$url}\{\textcolor{blue}{URL}\}\pause
\item \textcolor{violet}{$\backslash$href}\{\textcolor{blue}{URL}\}\{texto\}\pause
\end{itemize}

As opções definem-se dentro do comando: \textcolor{blue}{$\backslash$hypersetup}\{\ldots\}
\end{frame}
\begin{frame}
\frametitle{Opções do pacote hyperref}
{\setlength{\parskip}{0.2em}
\textcolor{blue}{\bfseries colorlinks=true}\\
\quad Links às cores (por omissão, ligações sublinhadas, sem cor).\pause

\textcolor{blue}{\bfseries linkcolor=cor}\\
\quad Cor das ligações internas (``blue'', ``green'', ``magenta'', \ldots).\pause

\textcolor{blue}{\bfseries citecolor=cor}\\
\quad Cor das citações.\pause

\textcolor{blue}{\bfseries urlcolor=cor}\\
\quad Cor das hiperligações.\pause

\textcolor{blue}{\bfseries pdftitle\{\ldots\}}\\
\quad Título do documento.\pause

\textcolor{blue}{\bfseries pdfauthor\{\ldots\}}\\
\quad Nome do autor.\pause

\textcolor{blue}{\bfseries pdfsubject\{\ldots\}}\\
\quad Assunto do documento.\pause

\textcolor{blue}{\bfseries pdfkeywords\{\ldots\}}\\
\quad Palavras chave, separadas por vírgulas.}
\end{frame}
\section{Blocos pré-formatados}
\begin{frame}
\frametitle{Blocos pré-formatados}
Usados para listagens de programas de computador ou outro texto que deva
ser apresentado textualmente.

\quad\textcolor{violet}{$\backslash$begin}\{\textcolor{blue}{verbatim}\}\\
\qquad \texttt{linha1}\\
\qquad \texttt{linha2}\\
\qquad\ldots\\
\quad\textcolor{violet}{$\backslash$end}\{\textcolor{blue}{verbatim}\}\pause

Dentro de um parágrafo pode também ser apresentada alguma parte de forma
textual, usando o comando \textcolor{blue}{$\backslash$verb}+\texttt{texto}+
\pause

O carácter usado para delimitar o texto (+ no caso acima), pode ser
qualquer que não apareça no texto.
\end{frame}
\section{Outros tipos de documentos}
\begin{frame}
\frametitle{Livros}
\quad\textcolor{violet}{$\backslash$documentclass}[\textcolor{brown}{opções}]\{\textcolor{blue}{book}\}\\
\quad\textcolor{violet}{$\backslash$begin}\{\textcolor{blue}{document}\}\\
\qquad\textcolor{brown}{$\backslash$frontmatter}\\
\qquad Capa, conteúdo, capítulo do prefácio, etc.\\
\qquad\textcolor{brown}{$\backslash$mainmatter}\\
\qquad\textcolor{violet}{$\backslash$part}\{Nome da parte\}\\
\qquad\textcolor{violet}{$\backslash$chapter}\{Título do capítulo\}\\
\qquad\textcolor{violet}{$\backslash$section}\{Título da secção\}\\
\qquad\textcolor{violet}{$\backslash$subsection}\{Título da subsecção\}\\
\qquad\textcolor{red}{$\backslash$appendix}\\
\qquad\textcolor{violet}{$\backslash$chapter}\{Título do apêndice\}\\
\qquad\textcolor{violet}{$\backslash$section}\{Título da secção\}\\
\qquad\textcolor{violet}{$\backslash$subsection}\{Título da subsecção\}\\
\qquad\textcolor{brown}{$\backslash$backmatter}\\
\qquad Bibliografia, índice remissivo, etc.\\
\quad\textcolor{violet}{$\backslash$end}\{\textcolor{blue}{document}\}
\end{frame}
\begin{frame}
\frametitle{Cartas}
\quad\textcolor{violet}{$\backslash$documentclass}[\textcolor{brown}{opções}]\{\textcolor{blue}{letter}\}\\
\quad\textcolor{violet}{$\backslash$address}\{Nome e morada do remetente\}\\
\quad\textcolor{violet}{$\backslash$signature}\{Nome de quem assina no fim\}\\
\quad\textcolor{violet}{$\backslash$date}\{data\}\\
\quad\textcolor{violet}{$\backslash$begin}\{\textcolor{blue}{document}\}\\
\qquad\textcolor{violet}{$\backslash$begin}\{\textcolor{blue}{letter}\}\{Nome e morada do destinatário\}\\
\qquad\textcolor{violet}{$\backslash$opening}\{Saudação\}\\
\qquad Texto da carta.\\
\qquad\textcolor{violet}{$\backslash$closing}\{Despedida\}\\
\qquad\textcolor{violet}{$\backslash$end}\{\textcolor{blue}{letter}\}\\
\quad\textcolor{violet}{$\backslash$end}\{\textcolor{blue}{document}\}
\end{frame}
\begin{frame}
\frametitle{Slides}
\quad\textcolor{violet}{$\backslash$documentclass}[\textcolor{brown}{opções}]\{\textcolor{blue}{beamer}\}\\
\quad\textcolor{violet}{$\backslash$usetheme}\{Nome do tema\}\\
\quad\textcolor{violet}{$\backslash$usecolor}\{Nome do esquema de cores\}\\
\quad\textcolor{violet}{$\backslash$begin}\{\textcolor{blue}{document}\}\\
\qquad\textcolor{violet}{$\backslash$begin}\{\textcolor{blue}{frame}\}\\
\qquad\textcolor{violet}{$\backslash$frametitle}\{Título do slide 1\}\\
\qquad Texto do slide 1.\\
\qquad\textcolor{violet}{$\backslash$end}\{\textcolor{blue}{frame}\}\\
\qquad\textcolor{violet}{$\backslash$begin}\{\textcolor{blue}{frame}\}\\
\qquad\textcolor{violet}{$\backslash$frametitle}\{Título do slide 2\}\\
\qquad Texto do slide 2.\\
\qquad\textcolor{violet}{$\backslash$end}\{\textcolor{blue}{frame}\}\\
\quad\textcolor{violet}{$\backslash$end}\{\textcolor{blue}{document}\}

O texto pode incluir pausas, usando o comando \textcolor{blue}{$\backslash$pause}
\end{frame}
\begin{frame}
\frametitle{Sítios úteis}
\begin{itemize}
\item Manual do Overlay: \url{https://www.overleaf.com/learn/}
\item Tabelas: \url{https://www.overleaf.com/learn/latex/Tables}
\item Equações:\\
  {\small \url{https://www.overleaf.com/learn/latex/Mathematical_expressions}}
\item Hiperligações:\\
  \url{https://www.overleaf.com/learn/latex/Hyperlinks}
\item Grupo de utilizadores de \TeX: \url{https://tug.org/}
\item Catálogo de fontes: \url{https://tug.org/FontCatalogue/}
\item Galeria de estilos e cores do \textbf{beamer}:\\
  {\small \url{https://deic-web.uab.cat/~iblanes/beamer_gallery/index.html}}
\end{itemize}
\end{frame}
\end{document}
